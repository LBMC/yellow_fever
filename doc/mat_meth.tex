\documentclass[a4paper,12pt]{article}

\begin{document}

The scripts used in this section are part of the scRNAtools R packages avalaible at :
The analysis itself using the scRNAtools package is available at :

\section{quality filtering \& normalization}

In this section we filter and correct for cell specific effect the counts table produced by Salmon.

We analysed first the *in vivo* data for the male donor (which are the most abundant).
To filter cells looking like blanks, we use the SVM-bagging algorithm
This algorithm run multiple SVM classifications between a group of cells labeled as blanks in the data and random sub-sample of unlabeled cells that should contain few cells looking like blanks \cite{mordeletBaggingSVMLearn2014b}.
We classify each time the full dataset from the fitted SVM and record the results.
We can then compute the probability of an unlabeled to look like a blank from the number of time that it has been classified as such.
Finally we run a last SVM classification trained on the 50\% of cells with the highest or lowest probability of looking like a blank to label the cells.
We removed from the subsequent analyses all the cells labeled as blank by the procedure.

We corrected for cells effect using the SCnorm procedure \cite{bacherSCnormRobustNormalization2017e}.
We processed each time-point independently to avoid removing any day specific effect.

With less cells for the female donor *in vivo* data, and for the *in vitro* data, we used the SVM model fitted on the male donor the classify these cells.
We removed from the subsequent analysis all the cells labeled as blank before correcting each time-point independently with SCnorm.


\section{cell-type classification \& differential expression analysis}

In this section we describe how we computed the probability for a given cells to be of memory type.

As a starting point of this classification we used

GNLY, GZMH, CCL4, KLRD1, GZMB and ZEB2 has representants of effector type genes.
LTB, TCF7, CCR7, GZMK and SELL has representants of memory type genes.
Finally the surface marker protein ccr7 and il7ra commonly used in the literature.



\cite{durifHighDimensionalClassification2018}
\cite{fournierADModelBuilder2012, skaugGeneralizedLinearMixed2016}


\bibliographystyle{abbrv}
\bibliography{bibliography}

\end{document}
